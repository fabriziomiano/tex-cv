% -*- program: xelatex -*-
%!TEX program = xelatex

%%%%%%%%%%%%%%%%%%%%%%%%%%%%%%%%%%%%%%%%%
% Friggeri Resume/CV
% XeLaTeX Template
% Version 1.2 (3/5/15)
%
% This template has been downloaded from:
% http://www.LaTeXTemplates.com
%
% Original author:
% Adrien Friggeri (adrien@friggeri.net)
% https://github.com/afriggeri/CV
%
% License:
% CC BY-NC-SA 3.0 (http://creativecommons.org/licenses/by-nc-sa/3.0/)
%
% Important notes:
% This template needs to be compiled with XeLaTeX and the bibliography, if used,
% needs to be compiled with biber rather than bibtex.
%
%%%%%%%%%%%%%%%%%%%%%%%%%%%%%%%%%%%%%%%%%

\documentclass[print]{cv} % Add 'print' as an option into the square bracket to remove colors from this template for printing
\addbibresource{bibliography.bib} % Specify the bibliography file to include publications

%----------------------------------------------------------------------------------------
%	CV INFO
%----------------------------------------------------------------------------------------
\cvname{Fabrizio Miano} % Your name
\cvjobtitle{Ph.D. in Fisica delle particelle} % Job title/career
\cvdate{26 Gennaio 1987} % Date of birth
\cvaddress{Brighton, UK} % Short address/location, use \newline if more than 1 line is required
\cvnumberphone{+447478953812} % Phone number
\cvlinkedin{\href{http://linkedin.com/in/fabriziomiano}{Profilo LinkedIn}} % LinkedIn 
\cvsussex{\href{http://www.sussex.ac.uk/profiles/356453}{Profilo Sussex Uni}} % LinkedIn 
\cvsite{\href{http://fabriziomiano.wordpress.com}{Personal Website}} % Personal website
\cvmail{\href{mailto:fabriziomiano@gmail.com}{fabriziomiano@gmail.com}} % Email address
\cvgit{\href{https://gitlab.com/fabriziomiano}{gitlab.com/fabriziomiano}}
%----------------------------------------------------------------------------------------
%	PROFILE PICTURE
%----------------------------------------------------------------------------------------
\newcommand{\profilepic}[1]{\renewcommand{\profilepic}{#1}}
\newlength\imagewidth
\newlength\imagescale
\pgfmathsetlength{\imagewidth}{4cm}
\pgfmathsetlength{\imagescale}{\imagewidth/600}
\profilepic{./figures/ID} % Profile picture


\begin{document}
\header{Fabrizio}{Miano}{\cvjobtitle} % Your name and current job title/field

%----------------------------------------------------------------------------------------
%	SIDEBAR SECTION
%----------------------------------------------------------------------------------------
% PROFILE PICTURE 
\begin{textblock}{0}(1, 0.6)
	% \begin{center}
		\begin{tikzpicture}[x=\imagescale,y=-\imagescale]
			\clip (300,300) circle (300);
			\node[anchor=north west, inner sep=0pt, outer sep=0pt] at (0,0) {\includegraphics[width=\imagewidth]{\profilepic}};
		\end{tikzpicture}
	% \end{center}
\end{textblock}

\begin{aside} % In the aside, each new line forces a line break
  \section{Contatti}
  % \begin{tabular}{cr}
  %   \Huge\icon{\textborn} & \hskip 0.5cm\cvdate\medskip
  %   \Large\icon{\Mundus}  & \hskip 0.5cm\cvaddress\medskip
  %   \Large\icon{\Telefon} & \hskip 0.5cm\cvnumberphone\medskip
  %   \Large\icon{\Info}    & \hskip 0.5cm\cvsite\medskip
  %   \Large\icon{\Letter}  & \hskip 0.5cm\href{mailto:\cvmail}{e-Mail}\medskip
  % \end{tabular}
    \cvmail
    \cvnumberphone
    \cvaddress
  ~
  \cvdate
  ~
  \section{Profili}
    \cvsite
    \cvlinkedin
    \cvsussex
    \cvgit
  % ~
  %   \cvdate
  ~
  \section{Lingue}
    Italiano Madrelingua
    Inglese Fluente
    Spagnolo Fluente
    Francese Elementare
  ~
   \section{Competenze tecniche} 
      Python, C++, HTML, 
      simulazioni Monte Carlo, 
      Bash, \LaTeX, Mac OS, UNIX, Windows, Office Suite
  ~
  \section{Volunteering}
    HiSPARC, 
    \#ScienceOnBuses,
    Brighton Science Festival,
    CERN Masterclass
\end{aside}


%----------------------------------------------------------------------------------------
%	WORK EXPERIENCE SECTION
%----------------------------------------------------------------------------------------

% \section{Competenze}
% Le principali abilità sviluppate durante il mio dottorato di ricerca riguardano il saper trattare con l'enorme volume di dati raccolti dall'esperimento ATLAS al Large Hadron Collider (LHC) del CERN (Ginevra). Nelle analisi dati su cui ho lavorato sono stati impiegati algoritmi per l'estrazione del segnale cercato, ottimizzazione del rapporto segnale su rumore (in Python) e modelling di regioni di controllo (in C++).

\section{Attività di ricerca in ATLAS}

\begin{entrylist}
  \entry
      {2014--2018}
      {University of Sussex e CERN}
      {Brighton, UK / Ginevra, Switzerland}
      {\emph{Posizione di dottorato}\\
      Supersimmetria ad LHC con l'esperimento ATLAS \\
      Dettagli: 
      \begin{itemize}
        \item Sviluppo di codice Python code per ottimizzazioni con multi variabili di rapporto segnale su fondo
        \item Modelling di fondi di Modello Standard utilizzando codice C++
        \item Stima Data-Driven di un fondo \emph{irriducibile} e rilevanti incertezze teoriche 
      \end{itemize}\medskip
      Performance del Inner Detector Trigger di ATLAS (Pixel detector, Semi-conductor Tracker, Transition Radiation Tracker)\\
      Dettagli: 
      \begin{itemize}
        \item Sviluppo di un algoritmo per la ricostruzione di vertici in tempo reale (C++)
        \item Algoritmi di estrazione di segnale fa milioni di eventi per secondo prodotti in collisioni tra particelle ad LHC 
        \item Sviluppo di codice HTML tramite una classe Python per il web monitoring della grandezza del Event Data Model di ATLAS
      \end{itemize}      
      }
      %------------------------------------------------
\end{entrylist}

\section{Attività di ricerca ad INFN-LNS}

\begin{entrylist}
  \entry
      {2013--2014}
      {Università degli Studi di Catania}
      {Catania, Italia}
      {\emph{MSc INFN-funded project}\\
      Simulazione Monte Carlo dell'emissione di protoni e caratterizzazione spazio-temporale in collisioni tra ioni pesanti 
      \begin{itemize}
        \item Sviluppo di un Monte Carlo (C++) per simulare l'emissione di protoni da una sorgente nucleare creata in collisioni tra ioni pesanti ad energie intermedie e 
        \item Validazione della simulazione tramite data/MC plot
        \item Deduzione della taglia geometrica e della vita media della sorgente utilizzando dati raccolti dal rivelatore LASSA del NSCL, Michigan, Stati Uniti. 
      \end{itemize}
      }

\end{entrylist}

%----------------------------------------------------------------------------------------
%	EDUCATION SECTION
%----------------------------------------------------------------------------------------
\section{Istruzione e Formazione}

\begin{entrylist}

  \entry
      {2014--2018}
      {PhD {\normalfont in Fisica delle Particelle}}
      {University of Sussex, Regno Unito / CERN, Svizzera}
      {\emph{Search for supersymmetry at LHC with ATLAS detector\\
      The performance of the Inner Detector Trigger of the ATLAS detector}
      }
      %------------------------------------------------

  \entry
      {2009--2014}
      {Laurea Magistrale {\normalfont in Fisica Nucleare}}
      {Università degli Studi di Catania, Italia}
      {\emph{Emissione di protoni e caratterizzazione spazio-temporale in collisioni tra ioni pesanti} }
      %------------------------------------------------

  \entry
      {2005--2009}
      {Laurea Triennale {\normalfont in Fisica}}
      {Università degli Studi di Catania, Italia}
      {\emph{Spettri di neutroni emessi da sorgenti standard Am-Be e Pu-Be} }
      %------------------------------------------------
\end{entrylist}

%----------------------------------------------------------------------------------------
% COMMUNICATION SKILLS SECTION
%----------------------------------------------------------------------------------------
\section{Competenze comunicative}

\begin{entrylist}

  \entry
    {2017}
    {Talk di 15 minuti}
    {Phenomenology 2017, Pittsburgh, PA, Stati Uniti}
    {\emph{Searches for direct production of third generation squarks with the ATLAS detector}}
    %------------------------------------------------
  \entry
    {2016}
    {Presentazione Poster}
    {ICHEP 2016, Chicago, IL, Stati Uniti}
    {\emph{The Design and performance of the ATLAS Inner Detector Trigger for Run 2 collisions at $\sqrt{s} = 13$ TeV}}
    %------------------------------------------------

  \entry
    {2015}
    {Presentazione Poster}
    {STFC HEP Summer School, Lancaster, Regno Unito}
    {\emph{Search for direct pair production of the top squark in all-hadronic final states in \emph{pp} collisions with the ATLAS detector}}
    %------------------------------------------------

  \entry
    {2014}
    {Talk di 20 minuti }
    {Università degli Studi di Catania, Italia}
    {\emph{Protons emission and space-time characterisation in heavy-ion collisions}}
    %------------------------------------------------
\end{entrylist}

%----------------------------------------------------------------------------------------
%	AWARDS SECTION
%----------------------------------------------------------------------------------------
\section{Premi}

\begin{entrylist}
  \entry
      {2016}
      {Grant}
      {University of Sussex, UK}
      {Doctoral Overseas Conference Grant for Postgraduate Researchers}
      %------------------------------------------------

  \entry
      {2015}
      {Borsa di studio}
      {STFC, UK}
      {4-year STFC-funded PhD scholarship in collaboration with CERN}
      %------------------------------------------------

  \entry
      {2013}
      {Borsa di studio}
      {INFN-LNS, Italia}
      {1 anno di fondi per Tesi magistrale in collaborazione con INFN-LNS}
      %------------------------------------------------
\end{entrylist}

%----------------------------------------------------------------------------------------
%	PUBLICATIONS SECTION
%----------------------------------------------------------------------------------------
 \section{Pubblicazioni}
 \printbibsection{article}{Articoli in giornali peer-reviewed} % Print all articles from the bibliography
 \begin{refsection} % This is a custom heading for those references marked as "inproceedings" but not containing "keyword=france"
 \nocite{*}
     \printbibliography[type=inproceedings, title={International peer-reviewed conferences/proceedings}, heading=bibheading]
\end{refsection}

\printbibsection{report}{Research reports} % Print all research reports from the bibliography

%----------------------------------------------------------------------------------------
\raggedbottom
\end{document}
