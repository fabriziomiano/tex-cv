% -*- program: xelatex -*-
%!TEX program = xelatex

%%%%%%%%%%%%%%%%%%%%%%%%%%%%%%%%%%%%%%%%%
% Friggeri Resume/CV
% XeLaTeX Template
% Version 1.2 (3/5/15)
%
% This template has been downloaded from:
% http://www.LaTeXTemplates.com
%
% Original author:
% Adrien Friggeri (adrien@friggeri.net)
% https://github.com/afriggeri/CV
%
% License:
% CC BY-NC-SA 3.0 (http://creativecommons.org/licenses/by-nc-sa/3.0/)
%
% Important notes:
% This template needs to be compiled with XeLaTeX and the bibliography, if used,
% needs to be compiled with biber rather than bibtex.
%
%%%%%%%%%%%%%%%%%%%%%%%%%%%%%%%%%%%%%%%%%

\documentclass[print]{cv} % Add 'print' as an option into the square bracket to remove colors from this template for printing
\addbibresource{bibliography.bib} % Specify the bibliography file to include publications

%----------------------------------------------------------------------------------------
%	CV INFO
%----------------------------------------------------------------------------------------
\cvname{Fabrizio Miano} % Your name
\cvjobtitle{Research Fellow in Data Science} % Job title/career
\cvdate{26 Gennaio 1987} % Date of birth
\cvaddress{Brighton, Regno Unito} % Short address/location, use \newline if more than 1 line is required
\cvnumberphone{+447478953812} % Phone number
\cvlinkedin{\href{http://linkedin.com/in/fabriziomiano}{linkedin.com/in/fabriziomiano}} % LinkedIn 
\cvsussex{\href{http://www.sussex.ac.uk/profiles/356453}{sussex.ac.uk/profiles/356453}} % LinkedIn 
\cvsite{\href{http://fabriziomiano.wordpress.com}{fabriziomiano.wordpress.com}} % Personal website
\cvmail{\href{mailto:fabriziomiano@gmail.com}{fabriziomiano@gmail.com}} % Email address
\cvgit{\href{https://gitlab.com/fabriziomiano}{gitlab.com/fabriziomiano}}
%----------------------------------------------------------------------------------------
%	PROFILE PICTURE
%----------------------------------------------------------------------------------------
\newcommand{\profilepic}[1]{\renewcommand{\profilepic}{#1}}
\newlength\imagewidth
\newlength\imagescale
\pgfmathsetlength{\imagewidth}{4cm}
\pgfmathsetlength{\imagescale}{\imagewidth/600}
\profilepic{./figures/ID} % Profile picture


\begin{document}
\header{Fabrizio}{Miano}{\cvjobtitle} % Your name and current job title/field

%----------------------------------------------------------------------------------------
%	SIDEBAR SECTION
%----------------------------------------------------------------------------------------
% PROFILE PICTURE 
\begin{textblock}{0}(1, 0.6)
	% \begin{center}
		\begin{tikzpicture}[x=\imagescale,y=-\imagescale]
			\clip (300,300) circle (300);
			\node[anchor=north west, inner sep=0pt, outer sep=0pt] at (0,0) {\includegraphics[width=\imagewidth]{\profilepic}};
		\end{tikzpicture}
	% \end{center}
\end{textblock}

\begin{aside} % In the aside, each new line forces a line break
  \section{Contatti}
  % \begin{tabular}{cr}
  %   \Huge\icon{\textborn} & \hskip 0.5cm\cvdate\medskip
  %   \Large\icon{\Mundus}  & \hskip 0.5cm\cvaddress\medskip
  %   \Large\icon{\Telefon} & \hskip 0.5cm\cvnumberphone\medskip
  %   \Large\icon{\Info}    & \hskip 0.5cm\cvsite\medskip
  %   \Large\icon{\Letter}  & \hskip 0.5cm\href{mailto:\cvmail}{e-Mail}\medskip
  % \end{tabular}
    \cvmail
    \cvnumberphone
    \cvaddress
  ~
  \cvdate
  ~
  \section{Profili}
    \cvsite
    \cvlinkedin
    \cvsussex
    \cvgit
  % ~
  %   \cvdate
  ~
  \section{Lingue}
    Italiano Madrelingua
    Inglese Fluente
    Spagnolo Fluente
    Francese Elementare
  ~
  \section{Computing} 
      Python, C++, HTML, Bash, \LaTeX,
      Machine Learning, 
      Monte Carlo simulations, 
      MS Word, Excel, PowerPoint,
      Mac OS, UNIX, Windows
  ~
  \section{Teaching}
    Associate Tutor of Classical Physics and Electromagnetism; Properties of Matter
  ~
  \section{Volontariato}
    HiSPARC, 
    \#ScienceOnBuses,
    Brighton Science Festival,
    CERN Masterclass
\end{aside}


%----------------------------------------------------------------------------------------
%	WORK EXPERIENCE SECTION
%----------------------------------------------------------------------------------------

\section{Overview}

Verso la fine del mio dottorato di ricerca ho vinto dei fondi per un progetto di Intelligenza Artificiale di 4 mesi in collaborazione con \href{http://www.deckchair.com}{Deckchair.com} riguardante lo sviluppo di un algortimo in grado di selezionare immagini interessanti all'interno di un dato dataset. I due approcci sviluppati sono: un primo filtro basato su Structural Similarity Index (OpenCV, Scikit-Image); un tool di machine learning che sfrutta Tensorflow, Keras e Scikit-Learn per la classificazione delle immagini. Il progetto è stato consegnato entro le deadline concordate. Oggi, in qualità di Research Fellow a University of Sussex mi occupo di \emph{action recognition} - la classificazione di un azione all'interno di un video - ed in particolare tramite di  di \emph{deception detection}. \\Durante il dottorato di ricerca ho avuto a che fare con enormi volumi di dati raccolti dall'esperimento ATLAS al Large Hadron Collider (LHC) del CERN (Ginevra). Nelle analisi dati sulle quali ho lavorato sono stati impiegati algoritmi per l'estrazione del segnale, ottimizzazione del rapporto segnale su rumore, e modelling di regioni di controllo.


\section{Attività di ricerca}

  \begin{entrylist}
    \entry
    {Ott -- Attuale}
    {University of Sussex}
    {Brighton, Regno Unito}
    {\emph{Research Fellow in Data Science}
    \begin{itemize}
      \item Sviluppo di un tool di AI (CNN, DNN) per deception detection (video/audio/testo)
      \item Creazione contatti con aziende potenzialmente interessate
      \item Esplorazione collaborazioni con altre aree di ricerca:
      \begin{itemize}
        \item Evolutionary and Adaptive Systems Research Group
        \item Data Science Research Group
        \item Industrial Informatics and Signal Processing Research Group
        \item Sensor Technology Research Centre
      \end{itemize}
    \end{itemize}
    }
  \end{entrylist}

  \begin{entrylist}
    \entry
    {2018}
    {University of Sussex in collaborazione con Deckchair.com}
    {Brighton, Regno Unito}
    {\emph{Junior Data Scientist}
    \begin{itemize}
      \item Sviluppo di un tool statistico in grado selezionare immagini ``interessanti'' all'interno di un dataset utilizzando \emph{Structural Similarity Index} (OpenCV, Scikit-Image);
      \item  Design di un classificatore d'immagini utilizzando K-Means clustering e Support Vector Machine (Tensorflow, Scikit-Learn)
    \end{itemize}
    }
  \end{entrylist}

  \begin{entrylist}
    \entry
    {2014--2018}
    {University of Sussex - CERN}
    {Brighton, Regno Unito / Ginevra, Svizzera}
    {\emph{Doctoral Position}
    \begin{itemize}
      \item Sviluppo codice Python code per analisi dati: ottimizzazione multi variabile del rapporto segnale/rumore per isolare regioni ricche di segnale
      \item Modelling delle principali sorgenti di rumore (C++)
      \item Stima Data-Driven di un rumore \emph{irreducibile} e valutazione delle incertezze teoriche associate ad esso (C++, Python, Excel)
      \item Partecipazione a sviluppo e commissioning di un algoritmo di ricostruzione di vertice primario d'interazione (C++)
      \item Validazione dei cambi al codice prima del rilascio
      \item Monitoring delle performance dell'efficienza di tracking
      \item Sviluppo di un tool (Python) di monitoring in grado di archiviare i risultati in un database (JSON) e visualizzarli su pagina web (HTML)
    \end{itemize}      
    }
  \end{entrylist}

  \begin{entrylist}
    \entry
    {2013--2014}
    {Università degli Studi di Catania}
    {Catania, Italia}
    {\emph{Progetto finanziato dal INFN}\\
      Simulazione Monte Carlo dell'emissione di protoni e caratterizzazione spazio-temporale in collisioni tra ioni pesanti 
      \begin{itemize}
        \item Sviluppo di un Monte Carlo (C++) per simulare l'emissione di protoni da una sorgente nucleare creata in collisioni tra ioni pesanti ad energie intermedie 
        \item Validazione della simulazione tramite data/MC plot
        \item Deduzione della taglia geometrica e della vita media della sorgente utilizzando dati raccolti dal rivelatore LASSA del NSCL, Michigan, Stati Uniti. 
      \end{itemize}
      }

\end{entrylist}

%----------------------------------------------------------------------------------------
%	EDUCATION SECTION
%----------------------------------------------------------------------------------------
\section{Istruzione e Formazione}

\begin{entrylist}

  \entry
      {2014--2018}
      {PhD {\normalfont in Fisica delle Particelle}}
      {University of Sussex, Regno Unito / CERN, Svizzera}
      {\emph{Search for supersymmetry at LHC with ATLAS detector\\
      The performance of the Inner Detector Trigger of the ATLAS detector}
      }
      %------------------------------------------------

  \entry
      {2009--2014}
      {Laurea Magistrale {\normalfont in Fisica Nucleare}}
      {Università degli Studi di Catania, Italia}
      {\emph{Emissione di protoni e caratterizzazione spazio-temporale in collisioni tra ioni pesanti} }
      %------------------------------------------------

  \entry
      {2005--2009}
      {Laurea Triennale {\normalfont in Fisica}}
      {Università degli Studi di Catania, Italia}
      {\emph{Spettri di neutroni emessi da sorgenti standard Am-Be e Pu-Be} }
      %------------------------------------------------
\end{entrylist}

%----------------------------------------------------------------------------------------
% COMMUNICATION SKILLS SECTION
%----------------------------------------------------------------------------------------
\section{Competenze comunicative}

\begin{entrylist}

  \entry
    {2017}
    {Talk di 15 minuti}
    {Phenomenology 2017, Pittsburgh, PA, Stati Uniti}
    {\emph{Searches for direct production of third generation squarks with the ATLAS detector}}
    %------------------------------------------------
  \entry
    {2016}
    {Presentazione Poster}
    {ICHEP 2016, Chicago, IL, Stati Uniti}
    {\emph{The Design and performance of the ATLAS Inner Detector Trigger for Run 2 collisions at $\sqrt{s} = 13$ TeV}}
    %------------------------------------------------

  \entry
    {2015}
    {Presentazione Poster}
    {STFC HEP Summer School, Lancaster, Regno Unito}
    {\emph{Search for direct pair production of the top squark in all-hadronic final states in \emph{pp} collisions with the ATLAS detector}}
    %------------------------------------------------

  \entry
    {2014}
    {Talk di 20 minuti}
    {Università degli Studi di Catania, Italia}
    {\emph{Protons emission and space-time characterisation in heavy-ion collisions}}
    %------------------------------------------------
\end{entrylist}

%----------------------------------------------------------------------------------------
%	AWARDS SECTION
%----------------------------------------------------------------------------------------
\section{Premi}

\begin{entrylist}
  \entry
      {2016}
      {Grant}
      {University of Sussex, Regno Unito}
      {Doctoral Overseas Conference Grant for Postgraduate Researchers}
      %------------------------------------------------

  \entry
      {2015}
      {Borsa di studio}
      {STFC, Regno Unito}
      {4-year STFC-funded PhD scholarship in collaboration with CERN}
      %------------------------------------------------

  \entry
      {2013}
      {Borsa di studio}
      {INFN-LNS, Italia}
      {1 anno di fondi per Tesi magistrale in collaborazione con INFN-LNS}
      %------------------------------------------------
\end{entrylist}

%----------------------------------------------------------------------------------------
%	PUBLICATIONS SECTION
%----------------------------------------------------------------------------------------
\newpage \section{Pubblicazioni}
 \printbibsection{article}{Articoli in giornali peer-reviewed} % Print all articles from the bibliography
 \begin{refsection} % This is a custom heading for those references marked as "inproceedings" but not containing "keyword=france"
 \nocite{*}
     \printbibliography[type=inproceedings, title={Proceedings in conferenze internazionali peer-reviewed}, heading=bibheading]
\end{refsection}

% \printbibsection{report}{Research reports} % Print all research reports from the bibliography

%----------------------------------------------------------------------------------------
\raggedbottom
\end{document}
