%----------------------------------------------------------------------------------------
% WORK EXPERIENCE SECTION
%----------------------------------------------------------------------------------------

% \cvsection{Experience}
\cvsection{Industrial Experience}
  \begin{cventries}
    \cventry
    {ICT Specialist}
    {R2M Solution}    
    {Catania, Italy}
    {Jan 2020 -- Now}
    {
      \begin{cvitems}
        \item {Full-stack development of internal management tools using python language and front-end technologies}
        \item {DevOps and deployment of web applications on Google cloud, AWS, Azure, and Heroku}
        \item {System administrator for internal tools and partner solutions}
        \item {Technical activities in research projects within the robotics, artificial intelligence, big data, energy saving, and IoT domains}
        \item {Delivery of high quality reports for research and innovation projects}
        \item {IT solutions and help desk}
      \end{cvitems}
    }

    \cventry
    {Data Scientist}
    {Leonardo}    
    {Catania, Italy}
    {Jan 2019 -- Now}
    {
      \begin{cvitems}
        \item {Data crawling: development of a Python tool to scrape LinkedIn jobs and user profiles}
        \item {Recommendation engine: development of a job-user recommendation system}
        \item {Computer Vision: development of a WebApp to perform real-time object detection using YOLO}
        \item {NLP: development of a deep-learning tool to extract the objects seized to criminal organizations from legal verdicts}
        \item {Social-network Analysis: named-entity extraction and sentiment analysis on Facebook-post comments}
        \item {Creation and management of micro-services on Azure}
      \end{cvitems}
    }

    \cventry
    {Junior Data Scientist}
    {Deckchair.com}
    {Brighton, UK}
    {Jun -- Aug 2018}
    {
      \begin{cvitems}
        \item Development of a statistical tool to select interesting images within a dataset;
        \item Design of a ML tool to classify grey sky and sunset images;
      \end{cvitems}
    }
  \end{cventries}

\cvsection{Academic Experience}
  \begin{cventries}
    \cventry
    {Post-Doctoral Research Fellow in Data Science}
    {University of Sussex}
    {Brighton, UK}
    {Sep -- Dec 2018}
    {
      \begin{cvitems}
        \item Development of ML model to classify videos of trials as Truthful/Deceptive
        \item Exploration of potential collaboration with other areas of research: 
        \begin{itemize}
          \item Evolutionary and Adaptive Systems Research Group
          \item Industrial IT and Signal Processing Research Group
          \item Sensor Technology Research Centre
        \end{itemize}
      \end{cvitems}
    }
  \end{cventries}

  \begin{cventries}    
    \cventry
    {Doctoral Position}
    {University of Sussex / CERN}
    {Brighton, UK / Geneva, Switzerland}
    {Nov 2014 -- Oct 2018}
    {
    \begin{cvitems}
      \item Developed Python code to perform multi-variable signal-over-background optimizations to define signal-enriched regions to enhance signal contribution while discriminating against background
      \item Modelling of main backgrounds using custom C++ framework
      \item Data-Driven estimation of an \emph{irreducible} background and evaluation of related theory uncertainties (C++, Python, Excel)
      \item Participation in the development and commissioning of vertex reconstruction algorithm (C++)
      \item Validation of the changes to the used framework before release
      \item Monitoring of the performance of the tracking efficiency
      \item Development of a monitoring tool (Python) to display the results of the monitoring (plots, tables) on a web page (HTML)
    \end{cvitems}
    }
  \end{cventries}

  \begin{cventries}    
    \cventry
    {MSc INFN-funded project}
    {Università degli Studi di Catania}
    {Catania, Italy}
    {Mar 2013 -- Jul 2014}
    {
    \begin{cvitems}
      \item Developed C++ code to simulate physical distributions of the emission of protons from a nuclear source of a given size and life-time
      \item Validation of the simulations
      \item Deduction of the geometrical size and life-time of a nuclear source created in a heavy-ion collision at intermediate energy using data collected with the LASSA detector of the NSCL, Michigan, US.
    \end{cvitems}
    }
  \end{cventries}
