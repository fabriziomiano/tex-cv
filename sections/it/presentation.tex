%-------------------------------------------------------------------------------
%	SECTION TITLE
%-------------------------------------------------------------------------------
% \cvsection{Presentation}
% %-------------------------------------------------------------------------------
% %	CONTENT
% %-------------------------------------------------------------------------------
% \begin{cventries}

% %---------------------------------------------------------
%   \cventry
%     {Presenter for <Hosting Web Application for Free utilizing GitHub, Netlify and CloudFlare>} % Role
%     {DevFest Seoul by Google Developer Group Korea} % Event
%     {Seoul, S.Korea} % Location
%     {Nov. 2017} % Date(s)
%     {
%       \begin{cvitems} % Description(s)
%         \item {Introduced the history of web technology and the JAM stack which is for the modern web application development.}
%         \item {Introduced how to freely host the web application with high performance utilizing global CDN services.}
%       \end{cvitems}
%     }

% %---------------------------------------------------------
% \end{cventries}



%----------------------------------------------------------------------------------------
% COMMUNICATION SKILLS SECTION
%----------------------------------------------------------------------------------------
\cvsection{Public Speaking}

\begin{cventries}
  \cventry
    {``From Particle Physics to Computer Vision''}
    {DISCnet showcase}
    {Royal Society, Londra, UK}
    {Dic 2018}
    {10-min talk sul software sviluppato per Deckchair.com}
    %------------------------------------------------

  \cventry
    {``HLT tracking performance''}
    {HLT-UK Meeting}
    {University of Oxford, UK}
    {Set 2017}
    {15-min talk: I risultati delle performance del tracking dell'Inner Detector di ATLAS }
    %------------------------------------------------

  \cventry
    {
    \href{http://cds.cern.ch/record/2263055/files/ATL-PHYS-SLIDE-2017-220.pdf}{``Searches for direct production of third generation squarks with the ATLAS detector''}
    }
    {Phenomenology 2017}
    {Pitgtsburgh, PA, Stati Uniti}
    {Mag 2017}
    {15-min talk: I risultati delle riceerche di squarks di terza generazione per conto della collaborazione ATLAS}
    %------------------------------------------------

  \cventry
    {
    \href{https://indico.cern.ch/event/432527/contributions/1071672/attachments/1317672/1974610/FMiano-IDTrigger-ICHEP.pdf}
    {``The Design and performance of the ATLAS Inner Detector Trigger for Run 2 collisions at $\sqrt{s} = 13$ TeV''}
    }
    {International Conference on High Energhy Physics (ICHEP) 2016}
    {Chicago, IL, Stati Uniti}
    {Aug 2016}
    {Poster presentato per conto della collaborazione ATLAS: ``Results of the performance of the ATLAS Inner Detector''}
    %------------------------------------------------

  \cventry
    {``Direct pair production of the top squark in all-hadronic final states in \emph{pp} collisions with the ATLAS detector''}
    {STFC HEP Summer School}
    {Lancaster, UK}
    {2015}
    {Poster: ``preliminary results of the optimisation of the regions of interest for the search of the supersymmetric partner of the top quark''}
    %------------------------------------------------

\end{cventries}