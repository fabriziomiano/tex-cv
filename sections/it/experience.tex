%----------------------------------------------------------------------------------------
% WORK EXPERIENCE SECTION
%----------------------------------------------------------------------------------------

\cvsection{Esperienza aziendale}
  \begin{cventries}
    \cventry
    {Senior DQ Analyst}
    {Coca Cola HBC}    
    {Remoto}
    {Mag 2023 -- Corrente}
    {
      \begin{cvitems}
        \item {Responsabile dell'implementazione e dell'esecuzione delle iniziative di data governance guidate dall'azienda,
        per mantenere alta la qualità della componente upstream della data chain;}
        \item{comprensione dei requisiti aziendali di data quality e traduzione in soluzioni tecniche;}
        \item{abbracciare il metodo agile: comprendere la natura iterativa dei progetti di dati e accogliere i cambiamenti per garantire un'elevata soddisfazione per gli utenti di dati a valle e collaborare con più parti dell'azienda e data users, inclusi data engineers, data scientists, team funzionali e data owners}
      \end{cvitems}
    }

    \cventry
    {Senior Consultant}
    {Capgemini}    
    {Remoto}
    {Mar 2021 -- Apr 2023}
    {Senior Data Engineer e Team Leader in un progetto internazionale con focus sullo sviluppo di una soluzione ETL end-to-end di business intelligence per l'analisi della customer demand e demand planning con metodologia Agile e stack MS Azure: Data Factory, Databricks, SQL Server, Synapse Analytics, Data Lake, Analysis Services e PowerBI; CI/CD con Jenkins e Terraform; code reviewer su Gerrit/Github/Azure Repos}

    \cventry
    {Business Intelligence Consultant}
    {Experis IT}    
    {Remoto}
    {Set 2020 -- Mar 2021}
    {Data Engineering consultancy per Capgemini su Microsoft Azure}

    \cventry
    {ICT Specialist / PM}
    {R2M Solution}    
    {Remoto}
    {Gen 2020 -- Set 2020}
    {
      \begin{cvitems}
        \item {Full-stack development di un ERP Custom (Frappé}
        \item {System administrator di web solutions on-premises e su cloud: Google cloud, AWS, Azure ed Heroku}
        \item {Attività tecniche in progetti di ricerca nei domini di robotica, intelligenza artificiale, big data, energy saving, ed IoT}
        \item {IT Service Management ed help desk}
      \end{cvitems}
    }

    \cventry
    {Data Scientist / Backend Developer}
    {Prisma}    
    {Catania, Italia}
    {Gen 2019 -- Gen 2020}
    {
      \begin{cvitems}
        \item {Data crawling: sviluppo di un crawler scritto in Python per la raccolta dati da social ne}
        \item {Sistemi di raccomandazione: sviluppo di un sistema di raccomandazione per user-job matching}
        \item {Computer Vision: sviluppo di webapp per real-time object detection usando YOLO}
        \item {NLP: sviluppo di un tool per l'estrazione di informazioni da PDF; integrazione con RESTful API (Python, Flask)}
        \item {Social-network Analysis: named-entity extraction e sentiment analysis su commenti a post su Facebook; integrazione con RESTful API (Python, Flask)}
        \item {Creazione e gestione microservizi su portale Azure (Docker ed Azure Container Registry)}
      \end{cvitems}
    }

    \cventry
    {Junior Data Scientist}
    {Deckchair.com}
    {Brighton, UK}
    {Giu -- Ago 2018}
    {
      \begin{cvitems}
        \item Sviluppo di software per la selezione di immagini interessanti all'interno di un dataset (Python, skimage);
        \item Design di AI per classificare immagini di cieli grigi e tramonti (Python, sklearn, Keras);
      \end{cvitems}
    }
  \end{cventries}


\cvsection{Esperienza Accademica}
  \begin{cventries}
    \cventry
    {Post-Doctoral Research Fellow in Data Science}
    {University of Sussex}
    {Brighton, UK}
    {Set -- Dic 2018}
    {
      \begin{cvitems}
        \item Sviluppo di modelli di machine learning per classificare video di testimonianze in tribunale come veritiere o false (Python, Keras, Tensorflow)
        \item Esplorazione di potenziali collaborazioni con altre aree di ricerca a University of Sussex quali, Evolutionary and Adaptive Systems Research Group, Industrial Informatics and Signal Processing Research Group, Sensor Technology Research Centre
      \end{cvitems}
    }
  \end{cventries}

  \begin{cventries}    
    \cventry
    {Dottorato di Ricerca}
    {University of Sussex / CERN}
    {Brighton, UK / Ginevra, Svizzera}
    {Nov 2014 -- Oct 2018}
    {
    \begin{cvitems}
      \item Sviluppo di codice Python per ottimizzazione multi-variabile signal-over-background 
      \item Modelling di diversi rumori di fondo, usando un custom C++ framework
      \item Stima data-driven di un rumore \emph{irriducibile} e valutazione delle incertezze teoriche (C++, Python, Excel)
      \item Partecipazione a sviluppo e commissioning di un algoritmo di ricostruzione del vertice della reazione (C++)
      \item Validazione pre-production di cambiamenti al framework
      \item Monitoring delle performance delle efficienze di tracking
      \item Sviluppo di codice Python per monitorare, e mostrare su una webpage (HTML), i risultati di nuove nightlies del codice.
    \end{cvitems}
    }
  \end{cventries}

  \begin{cventries}    
    \cventry
    {Progetto di ricerca - Laurea Magistrale}
    {Università degli Studi di Catania}
    {Catania, Italia}
    {Mar 2013 -- Jul 2014}
    {
    \begin{cvitems}
      \item Sviluppo e validazione codice C per simulare distribuzioni d'interesse fisico dell'emissione di protoni da una sorgente nucleare di data taglia geometrica e vita media. 
      \item Deduzione di taglia geometrica e vita media di una sorgente nucleare creata in collisioni tra ioni pesanti ad energie intermedie, usando dati del rivelatore LASSA del NSCL, Michigan, Stati Uniti.
    \end{cvitems}
    }
  \end{cventries}