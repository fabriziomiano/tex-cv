%----------------------------------------------------------------------------------------
% WORK EXPERIENCE SECTION
%----------------------------------------------------------------------------------------

% \cvsection{Experience}
\cvsection{Esperienza aziendale}
  \begin{cventries}
    \cventry
    {Data Scientist - Sviluppo AI}
    {Leonardo}    
    {Catania, Italia}
    {Gen 2019 -- Corrente}
    {
      \begin{cvitems}
        \item {Data crawling: sviluppo in Python di uno strumento per la raccolta dati da LinkedIn: offerte di lavoro e profili utente}
        \item {Sistemi di raccomandazione: sviluppo di un sistema di raccomandazione per user-job matching}
        \item {Computer Vision: sviluppo di webapp per real-time object detection usando YOLO}
        \item {NLP: sviluppo di un tool per l'estrazione di oggetti confiscati alle mafie da sentenze penali}
        \item {Social-network Analysis: named-entity extraction e sentiment analysis su commenti a post su Facebook}
        \item {Creazione e gestione microservizi su portale Azure}
      \end{cvitems}
    }

    \cventry
    {Junior Data Scientist}
    {Deckchair.com}
    {Brighton, UK}
    {Giu -- Ago 2018}
    {
      \begin{cvitems}
        \item Sviluppo di software per la selezione di immagini interessanti all'interno di un dataset
        \item Design di AI per classificare immagini di cieli grigi e tramonti;
      \end{cvitems}
    }
  \end{cventries}

\cvsection{Esperienza Accademica}
  \begin{cventries}
    \cventry
    {Post-Doctoral Research Fellow in Data Science}
    {University of Sussex}
    {Brighton, UK}
    {Set -- Dic 2018}
    {
      \begin{cvitems}
        \item Sviluppo di modelli di machine learning per classificare video di testimonianze in tribunale come veritiere o false
        \item Esplorazione di potenziali collaborazioni con altre aree di ricerca a University of Sussex: 
        \begin{itemize}
          \item Evolutionary and Adaptive Systems Research Group
          \item Industrial Informatics and Signal Processing Research Group
          \item Sensor Technology Research Centre
        \end{itemize}
      \end{cvitems}
    }
  \end{cventries}

  \begin{cventries}    
    \cventry
    {Dottorato di Ricerca}
    {University of Sussex / CERN}
    {Brighton, UK / Ginevra, Svizzera}
    {Nov 2014 -- Oct 2018}
    {
    \begin{cvitems}
      \item Sviluppo di codice Python per ottimizzazione multi-variabile signal-over-background 
      \item Modelling di diversi rumori di fondo, usando un custom C++ framework
      \item Stima data-driven di un rumore \emph{irriducibile} e valutazione delle incertezze teoriche (C++, Python, Excel)
      \item Partecipazione a sviluppo e commissioning di un algoritmo di ricostruzione del vertice della reazione (C++)
      \item Validazione pre-production di cambiamenti al framework
      \item Monitoring delle performance delle efficienze di tracking
      \item Sviluppo di codice Python per monitorare, e mostrare su una webpage (HTML), i risultati di nuove nightlies del codice.
    \end{cvitems}
    }
  \end{cventries}

  \begin{cventries}    
    \cventry
    {Progetto di ricerca - Laurea Magistrale}
    {Università degli Studi di Catania}
    {Catania, Italy}
    {Mar 2013 -- Jul 2014}
    {
    \begin{cvitems}
      \item Sviluppo e validazione codice C per simulare distribuzioni d'interesse fisico dell'emissione di protoni da una sorgente nucleare di data taglia geometrica e vita media. 
      \item Deduzione di taglia geometrica e vita media di una sorgente nucleare creata in collisioni tra ioni pesanti ad energie intermedie, usando dati del rivelatore LASSA del NSCL, Michigan, Stati Uniti.
    \end{cvitems}
    }
  \end{cventries}
