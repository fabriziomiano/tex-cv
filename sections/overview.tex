\cvsection{Overview}

I am a Data Scientist at Leonardo working on a number of AI projects within the scopes of energy market, public administration, and security, mainly focusing on computer vision, text mining, and natural language processing. Previously, as a Research Fellow at the University of Sussex, I was focusing on the development of an AI capable of classifying video of trials as truthful/deceptive. I did a PhD in Experimental Particle Physics at the University of Sussex where I developed skills such as, problem-solving, time management, and especially how to deal with high-rates and high-volumes of data from particle collisions collected with the ATLAS detector at CERN, Geneva. 
% Signal extraction algorithms, signal-over-background optimisations (Python) modelling of regions of control (C\textit{++}), and estimation of uncertainties (MS Excel) were at the heart of the analyses I have carried out. Ultimately, I have developed a monitoring tool (Python) for the ATLAS Inner Detector Trigger that employs the creation and the management of a database.
% Previously, I joined a placement programme as a Junior Data Scientist at \href{http://www.deckchair.com}{Deckchair.com}. Here, I developed an algorithm capable of selecting interesting images within a given dataset, by exploiting the structural similarity index, and a machine learning tool for image classification. 


% By the end of my PhD I won a grant to work on an AI project: the development of an algorithm capable of selecting interesting images within a given dataset. I have spent 4 months working for \href{http://www.deckchair.com}{Deckchair.com} developing such an algorithm. In particular, two of the approaches that have been developed are; a statistical tool that exploits the structural similarity index (OpenCV, skimage); a machine learning tool (Scikit-Learn, keras, Tensorflow) for image classification. The project was timely delivered. As a Research Fellow in Data Science, I have moved to a new project: action recognition i.e. the classification of an action that is present in a given video. \\During my PhD the main skills I have developed are related to dealing with high-rates and high-volumes of particle-particle collisions data collected with the ATLAS detector at CERN, Geneva. Signal extraction algorithms, signal-over-background optimisations (Python) modelling of regions of control (C\textit{++}), and estimation of uncertainties (MS Excel) were at the heart of the analyses I have carried out. Ultimately, I have developed a monitoring tool (Python) for the ATLAS Inner Detector Trigger that employs the creation and the management of a database.