% -*- program: xelatex -*-
%!TEX program = xelatex

%%%%%%%%%%%%%%%%%%%%%%%%%%%%%%%%%%%%%%%%%
% Friggeri Resume/CV
% XeLaTeX Template
% Version 1.2 (3/5/15)
%
% This template has been downloaded from:
% http://www.LaTeXTemplates.com
%
% Original author:
% Adrien Friggeri (adrien@friggeri.net)
% https://github.com/afriggeri/CV
%
% License:
% CC BY-NC-SA 3.0 (http://creativecommons.org/licenses/by-nc-sa/3.0/)
%
% Important notes:
% This template needs to be compiled with XeLaTeX and the bibliography, if used,
% needs to be compiled with biber rather than bibtex.
%
%%%%%%%%%%%%%%%%%%%%%%%%%%%%%%%%%%%%%%%%%

\documentclass[print]{cv} % Add 'print' as an option into the square bracket to remove colors from this template for printing
\addbibresource{bibliography.bib} % Specify the bibliography file to include publications

%----------------------------------------------------------------------------------------
%	CV INFO
%----------------------------------------------------------------------------------------
\cvname{Fabrizio Miano} % Your name
\cvjobtitle{Ph.D. candidate in Experimental Particle Physics} % Job title/career
\cvdate{26 January 1987} % Date of birth
\cvaddress{Brighton, UK} % Short address/location, use \newline if more than 1 line is required
\cvnumberphone{+447478953812} % Phone number
\cvlinkedin{\href{http://linkedin.com/in/fabriziomiano}{LinkedIn Profile}} % LinkedIn 
\cvsussex{\href{http://www.sussex.ac.uk/profiles/356453}{Sussex Uni Profile}} % LinkedIn 
\cvsite{\href{http://fabriziomiano.wordpress.com}{Personal Website}} % Personal website
\cvmail{\href{mailto:fabriziomiano@gmail.com}{fabriziomiano@gmail.com}} % Email address
\cvgit{\href{https://github.com/fabriziomiano}{github.com/fabriziomiano}}
%----------------------------------------------------------------------------------------
%	PROFILE PICTURE
%----------------------------------------------------------------------------------------
\newcommand{\profilepic}[1]{\renewcommand{\profilepic}{#1}}
\newlength\imagewidth
\newlength\imagescale
\pgfmathsetlength{\imagewidth}{4cm}
\pgfmathsetlength{\imagescale}{\imagewidth/600}
\profilepic{./figures/ID} % Profile picture


\begin{document}
\header{Fabrizio}{Miano}{\cvjobtitle} % Your name and current job title/field

%----------------------------------------------------------------------------------------
%	SIDEBAR SECTION
%----------------------------------------------------------------------------------------
% PROFILE PICTURE 
% \begin{textblock}{0}(1, 0.5)
% 	% \begin{center}
% 		\begin{tikzpicture}[x=\imagescale,y=-\imagescale]
% 			\clip (300,300) circle (300);
% 			\node[anchor=north west, inner sep=0pt, outer sep=0pt] at (0,0) {\includegraphics[width=\imagewidth]{\profilepic}};
% 		\end{tikzpicture}
% 	% \end{center}
% \end{textblock}

\begin{aside} % In the aside, each new line forces a line break
  \section{Contacts}
  % \begin{tabular}{cr}
  %   \Huge\icon{\textborn} & \hskip 0.5cm\cvdate\medskip
  %   \Large\icon{\Mundus}  & \hskip 0.5cm\cvaddress\medskip
  %   \Large\icon{\Telefon} & \hskip 0.5cm\cvnumberphone\medskip
  %   \Large\icon{\Info}    & \hskip 0.5cm\cvsite\medskip
  %   \Large\icon{\Letter}  & \hskip 0.5cm\href{mailto:\cvmail}{e-Mail}\medskip
  % \end{tabular}
    \cvmail
    \cvnumberphone
    \cvaddress
  ~
  \section{Profiles}
    \cvsite
    \cvlinkedin
    \cvsussex
    \cvgit
  % ~
  %   \cvdate
  ~
  \section{Computing} 
      Python, C++, HTML, 
      Bash, Monte Carlo simulations, 
      \LaTeX, Mac OS, UNIX, Windows, Office Suite
  ~
  \section{Volunteering}
    HiSPARC, 
    \#ScienceOnBuses,
    Brighton Science Festival,
    CERN Masterclass
  ~
  \section{Teaching}
    Associate Tutor of Classical Physics and Electromagnetism; Properties of Matter
  ~
  \section{Languages}
    Italian Mother tongue
    English Fluent
    Spanish Fluent
    French Basic
  ~
\end{aside}


%----------------------------------------------------------------------------------------
%	WORK EXPERIENCE SECTION
%----------------------------------------------------------------------------------------

% \section{Professional Skills}
% The main skills I have developed during my PhD are related to dealing with high-rate and high-volume of data collected with the ATLAS detector at CERN, Geneva.
% Signal extraction algorithms were employed in the analyses I have worked on. Signal-over-background optimisations (Python), modelling of regions of control (C\textit{++}), and monitoring of the ATLAS Inner Detector Trigger were also the heart of such analyses. 

% \section{Experience}

  \subsection{Research at the ATLAS Experiment}
  \begin{entrylist}
    \entry
        {2014--Now}
        {University of Sussex and CERN}
        {Brighton, UK / Geneva, Switzerland}
        {\emph{Doctoral Position} - thesis submission expected early September 2018\\
        Supersymmetry searches at the LHC with the ATLAS detector\\
        Detailed involvement: 
        \begin{itemize}
          \item Developed Python code to perform multi-variable signal-over-background optimisations to define new kinematic signal regions that enhanced signal contribution while discriminating against SM background processes
          \item Modelling of Standard Model backgrounds (\ttbar, \Wjets, \Zjets) using custom C++ framework for ROOT n-tuple production to define control regions for the main backgrounds 
          \item Data-Driven estimation of the \emph{irreducible} \ttZ\ background and evaluation of related theory uncertainties
        \end{itemize}\medskip
        Performance of the ATLAS Inner Detector Trigger\\
        Detailed involvement: 
        \begin{itemize}
          \item Participation in the development of vertex reconstruction algorithm (C++) within the \emph{TrigInDetAnalysis} framework
          \item Validation of the changes to the \emph{TrigInDetAnalysis} framework before release
          \item Monitoring of the performance of electron, muon, and $b$-jet tracking and reconstruction efficiencies.
        \end{itemize}\medskip
        Development of the Trigger EDM Size Monitoring Python tool\\
        Detailed involvement: 
        \begin{itemize}
          \item Development of a Python class to monitor the size of the ATLAS Event Data Model for Run 2, and of HTML code to display the results on a web page
        \end{itemize}      
        }
        %------------------------------------------------
  \end{entrylist}

  \subsection{Research at the INFN-LNS}

  \begin{entrylist}
    \entry
        {2013--2014}
        {Università degli Studi di Catania}
        {Catania, Italy}
        {\emph{MSc INFN-funded project}\\
        Monte Carlo simulations of protons emission and space-time characterisation in heavy-ion collisions
        \begin{itemize}
          \item Developed C++ code to simulate physical distributions of the emission of protons from a nuclear source of a given size and life-time
          \item Validation of the simulations were performed by means of data/MC plots 
          \item Deduction of the geometrical size and life-time of a nuclear source created in a heavy-ion collision at intermediate energy using data collected with the LASSA detector of the NSCL, Michigan, US.
        \end{itemize}
        }

  \end{entrylist}

%----------------------------------------------------------------------------------------
%	EDUCATION SECTION
%----------------------------------------------------------------------------------------
\section{Education}

\begin{entrylist}
  \entry
      {Current}
      {PhD {\normalfont in Experimental Particle Physics}}
      {University of Sussex, UK / CERN, Switzerland}
      {\emph{Search for supersymmetry at LHC with ATLAS detector\\
      The performance of the Inner Detector Trigger of the ATLAS detector}
      }
      %------------------------------------------------

  \entry
      {2009--2014}
      {MSc {\normalfont in Experimental Nuclear Physics}}
      {Università degli Studi di Catania, Italy}
      {\emph{Protons emission and space-time characterisation in heavy-ion collisions} }
      %------------------------------------------------

  \entry
      {2005--2009}
      {BSc {\normalfont in Physics}}
      {Università degli Studi di Catania, Italy}
      {\emph{Study of neutrons spectra emitted by Am-Be and Pu-Be radioactive sources} }
      %------------------------------------------------
\end{entrylist}

%----------------------------------------------------------------------------------------
%	PUBLICATIONS SECTION
%----------------------------------------------------------------------------------------
 \clearpage
 \section{Publications}
 \printbibsection{article}{Articles in peer-reviewed journals} % Print all articles from the bibliography
 \begin{refsection} % This is a custom heading for those references marked as "inproceedings" but not containing "keyword=france"
 \nocite{*}
     \printbibliography[type=inproceedings, title={International peer-reviewed conferences/proceedings}, heading=bibheading]
\end{refsection}

%\printbibsection{report}{Research reports} % Print all research reports from the bibliography


%----------------------------------------------------------------------------------------
% COMMUNICATION SKILLS SECTION
%----------------------------------------------------------------------------------------
\section{Communication}

\begin{entrylist}
  \entry
    {2017}
    {15' Conference Talk}
    {Phenomenology 2017, Pittsburgh, PA, USA}
    {\emph{Searches for direct production of third generation squarks with the ATLAS detector}}
    %------------------------------------------------
  \entry
    {2016}
    {Conference Poster}
    {ICHEP 2016, Chicago, IL, USA}
    {\emph{The Design and performance of the ATLAS Inner Detector Trigger for Run 2 collisions at $\sqrt{s} = 13$ TeV}}
    %------------------------------------------------

  \entry
    {2015}
    {Poster}
    {STFC HEP Summer School, Lancaster, UK}
    {\emph{Search for direct pair production of the top squark in all-hadronic final states in \emph{pp} collisions with the ATLAS detector}}
    %------------------------------------------------

  \entry
    {2014}
    {20' Graduation Talk}
    {Università degli Studi di Catania, Italy}
    {\emph{Protons emission and space-time characterisation in heavy-ion collisions}}
    %------------------------------------------------
\end{entrylist}

%----------------------------------------------------------------------------------------
% AWARDS SECTION
%----------------------------------------------------------------------------------------
\section{Awards}

\begin{entrylist}
  \entry
      {2016}
      {Grant}
      {University of Sussex, UK}
      {Doctoral Overseas Conference Grant for Postgraduate Researchers}
      %------------------------------------------------

  \entry
      {2015}
      {Scholarship}
      {STFC, UK}
      {4-years STFC-funded PhD scholarship in collaboration with CERN}
      %------------------------------------------------

  \entry
      {2013}
      {Scholarship}
      {INFN-LNS, Italy}
      {1 year INFN-funded scholarship in collaboration with LNS}
      %------------------------------------------------
\end{entrylist}



\raggedbottom
\end{document}
