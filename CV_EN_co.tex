% -*- program: xelatex -*-
%!TEX program = xelatex

%%%%%%%%%%%%%%%%%%%%%%%%%%%%%%%%%%%%%%%%%
% Friggeri Resume/CV
% XeLaTeX Template
% Version 1.2 (3/5/15)
%
% This template has been downloaded from:
% http://www.LaTeXTemplates.com
%
% Original author:
% Adrien Friggeri (adrien@friggeri.net)
% https://github.com/afriggeri/CV
%
% License:
% CC BY-NC-SA 3.0 (http://creativecommons.org/licenses/by-nc-sa/3.0/)
%
% Important notes:
% This template needs to be compiled with XeLaTeX and the bibliography, if used,
% needs to be compiled with biber rather than bibtex.
%
%%%%%%%%%%%%%%%%%%%%%%%%%%%%%%%%%%%%%%%%%

\documentclass[print]{cv} % Add 'print' as an option into the square bracket to remove colors from this template for printing
\addbibresource{bibliography.bib} % Specify the bibliography file to include publications

%----------------------------------------------------------------------------------------
%	CV INFO
%----------------------------------------------------------------------------------------
\cvname{Fabrizio Miano} % Your name
\cvjobtitle{Research Fellow in Data Science} % Job title/career
\cvdate{26 January 1987} % Date of birth
\cvaddress{Brighton, UK} % Short address/location, use \newline if more than 1 line is required
\cvnumberphone{+447478953812} % Phone number
% \cvlinkedin{\href{http://linkedin.com/in/fabriziomiano}{linkedin.com/in/fabriziomiano}} 
\cvlinkedin{\href{http://linkedin.com/in/fabriziomiano}{linkedin.com/in/fabriziomiano}} 
% \cvsussex{\href{http://www.sussex.ac.uk/profiles/356453}{sussex.ac.uk/profiles/356453}} 
\cvsussex{\href{http://www.sussex.ac.uk/profiles/356453}{sussex.ac.uk/profiles/356453}}  
\cvsite{\href{http://fabriziomiano.wordpress.com}{fabriziomiano.wordpress.com}} 
\cvmail{\href{mailto:fabriziomiano@gmail.com}{fabriziomiano@gmail.com}} 
\cvgit{\href{https://gitlab.com/fabriziomiano}{gitlab.com/fabriziomiano}}
%----------------------------------------------------------------------------------------
%	PROFILE PICTURE
%----------------------------------------------------------------------------------------
\newcommand{\profilepic}[1]{\renewcommand{\profilepic}{#1}}
\newlength\imagewidth
\newlength\imagescale
\pgfmathsetlength{\imagewidth}{4cm}
\pgfmathsetlength{\imagescale}{\imagewidth/600}
\profilepic{./figures/ID} % Profile picture


\begin{document}
\header{Fabrizio}{Miano}{\cvjobtitle} % Your name and current job title/field

%----------------------------------------------------------------------------------------
%	SIDEBAR SECTION
%----------------------------------------------------------------------------------------
% PROFILE PICTURE 
% \begin{textblock}{0}(1, 0.5)
% 	% \begin{center}
% 		\begin{tikzpicture}[x=\imagescale,y=-\imagescale]
% 			\clip (300,300) circle (300);
% 			\node[anchor=north west, inner sep=0pt, outer sep=0pt] at (0,0) {\includegraphics[width=\imagewidth]{\profilepic}};
% 		\end{tikzpicture}
% 	% \end{center}
% \end{textblock}

\begin{aside} % In the aside, each new line forces a line break
  \section{Contacts}
  % \begin{tabular}{cr}
  %   \Huge\icon{\textborn} & \hskip 0.5cm\cvdate\medskip
  %   \Large\icon{\Mundus}  & \hskip 0.5cm\cvaddress\medskip
  %   \Large\icon{\Telefon} & \hskip 0.5cm\cvnumberphone\medskip
  %   \Large\icon{\Info}    & \hskip 0.5cm\cvsite\medskip
  %   \Large\icon{\Letter}  & \hskip 0.5cm\href{mailto:\cvmail}{e-Mail}\medskip
  % \end{tabular}
    \cvmail
    \cvnumberphone
    \cvaddress
  ~
  \section{Profiles}
    \cvlinkedin
    \cvsussex
    \cvgit
    \cvsite
  % ~
  %   \cvdate
  ~
  \section{Computing} 
      Python, C++, HTML, Bash, \LaTeX,
      Machine Learning, 
      Monte Carlo simulations, 
      MS Word, Excel, PowerPoint,
      Mac OS, UNIX, Windows
  ~
  \section{Teaching}
    Associate Tutor of Classical Physics and Electromagnetism; Properties of Matter
  ~
  \section{Volunteering}
    HiSPARC, 
    \#ScienceOnBuses,
    Brighton Science Festival,
    CERN Master class
  ~
  \section{Languages}
    Italian Mother tongue
    English Fluent
    Spanish Fluent
    French Basic
  ~
\end{aside}


%----------------------------------------------------------------------------------------
%	WORK EXPERIENCE SECTION
%----------------------------------------------------------------------------------------

\section{Overview}

By the end of my PhD I won a grant to work on an AI project: the development of an algorithm capable of selecting interesting images within a given dataset. I have spent 4 months working for \href{http://www.deckchair.com}{Deckchair.com} developing such an algorithm. In particular, two of the approaches that have been developed are; a statistical tool that exploits the structural similarity index (OpenCV, skimage); a machine learning tool (Scikit-Learn, keras, Tensorflow) for image classification. The project was timely delivered. As a Research Fellow in Data Science, I have moved to a new project: action recognition i.e. the classification of an action that is present in a given video. \\During my PhD the main skills I have developed are related to dealing with high-rates and high-volumes of particle-particle collisions data collected with the ATLAS detector at CERN, Geneva. Signal extraction algorithms, signal-over-background optimisations (Python) modelling of regions of control (C\textit{++}), and estimation of uncertainties (MS Excel) were at the heart of the analyses I have carried out. Ultimately, I have developed a monitoring tool (Python) for the ATLAS Inner Detector Trigger that employs the creation and the management of a database.

 \section{Research Experience}
  \begin{entrylist}
    \entry
    {2018--Now}
    {University of Sussex}
    {Brighton, UK}
    {\emph{Research Fellow in Data Science}
    \begin{itemize}
      \item Development of an AI tool capable of performing action recognition (neural networks: CNN, DNN)
      \item Seek companies to work with on specific challenges
      \item Exploration of potential collaboration with other areas of research: 
      \begin{itemize}
        \item Evolutionary and Adaptive Systems Research Group
        \item Data Science Research Group
        \item Industrial Informatics and Signal Processing Research Group
        \item Sensor Technology Research Centre
      \end{itemize}
    \end{itemize}
    }
  \end{entrylist}

  \begin{entrylist}
    \entry
    {2018}
    {University of Sussex}
    {Brighton, UK}
    {\emph{Junior Data Scientist - Placement in collaboration with Deckchair.com}
    \\Detailed involvement:
    \begin{itemize}
      \item Development of a statistical tool to detect interesting images within a dataset using Structural Similarity Index analysis (OpenCV, Scikit-Learn);
      \item Design of a machine learning tool to perform feature extraction via k-Means clustering (Scikit-Learn) and image classification using SVM, Keras, and Tensorflow (Inception V3);
    \end{itemize}
    }
  \end{entrylist}

  \begin{entrylist}
    \entry
    {2014--2018}
    {University of Sussex - CERN}
    {Brighton, UK / Geneva, Switzerland}
    {\emph{Doctoral Position}
    \begin{itemize}
      \item Developed Python code to perform multi-variable signal-over-background optimisations to define signal-enriched regions to enhance signal contribution while discriminating against background
      \item Modelling of main backgrounds using custom C++ framework
      \item Data-Driven estimation of an \emph{irreducible} background and evaluation of related theory uncertainties (C++, Python, Excel)
    \end{itemize}%\medskip
    % Performance of the ATLAS Inner Detector Trigger\\
    % Detailed involvement: 
    \begin{itemize}
      \item Participation in the development and commissioning of vertex reconstruction algorithm (C++)
      \item Validation of the changes to the used framework before release
      \item Monitoring of the performance of the tracking efficiencies
    \end{itemize}%\medskip
    % Development of the Trigger EDM Size Monitoring Python tool\\
    % Detailed involvement: 
    \begin{itemize}
      \item Development of a monitoring tool (Python) to display the results of the monitoring (plots, tables) on a web page (HTML)
    \end{itemize}      
    }
  \end{entrylist}

  \begin{entrylist}
    \entry
    {2013--2014}
    {Università degli Studi di Catania}
    {Catania, Italy}
    {\emph{MSc INFN-funded project}\\
    Monte Carlo simulations of protons emission and space-time characterisation in heavy-ion collisions
    \begin{itemize}
      \item Developed C++ code to simulate physical distributions of the emission of protons from a nuclear source of a given size and life-time
      \item Validation of the simulations
      \item Deduction of the geometrical size and life-time of a nuclear source created in a heavy-ion collision at intermediate energy using data collected with the LASSA detector of the NSCL, Michigan, US.
    \end{itemize}
    }
  \end{entrylist}

%----------------------------------------------------------------------------------------
%	EDUCATION SECTION
%----------------------------------------------------------------------------------------
\section{Education}

\begin{entrylist}
  \entry
      {2015--2018}
      {PhD {\normalfont in Experimental Particle Physics}}
      {University of Sussex, UK / CERN, Switzerland}
      {\emph{Optimisation studies and data-driven background estimation in searches for the 
      supersymmetric partner of the top quark with the ATLAS Detector at the LHC}
      }
      %------------------------------------------------

  \entry
      {2009--2014}
      {MSc {\normalfont in Experimental Nuclear Physics}}
      {Università degli Studi di Catania, Italy}
      {\emph{Protons emission and space-time characterisation in heavy-ion collisions} }
      %------------------------------------------------

  \entry
      {2005--2009}
      {BSc {\normalfont in Physics}}
      {Università degli Studi di Catania, Italy}
      {\emph{Study of neutrons spectra emitted by Am-Be and Pu-Be radioactive sources}}
      %------------------------------------------------
\end{entrylist}



%----------------------------------------------------------------------------------------
% COMMUNICATION SKILLS SECTION
%----------------------------------------------------------------------------------------
\section{Communication Skills}

\begin{entrylist}
  \entry
    {2017}
    {15' Conference Talk}
    {Phenomenology 2017, Pittsburgh, PA, USA}
    {\emph{Searches for direct production of third generation squarks with the ATLAS detector}}
    %------------------------------------------------
  \entry
    {2016}
    {Conference Poster}
    {ICHEP 2016, Chicago, IL, USA}
    {\emph{The Design and performance of the ATLAS Inner Detector Trigger for Run 2 collisions at $\sqrt{s} = 13$ TeV}}
    %------------------------------------------------

  \entry
    {2015}
    {Poster}
    {STFC HEP Summer School, Lancaster, UK}
    {\emph{Search for direct pair production of the top squark in all-hadronic final states in \emph{pp} collisions with the ATLAS detector}}
    %------------------------------------------------

  \entry
    {2014}
    {20' Graduation Talk}
    {Università degli Studi di Catania, Italy}
    {\emph{Emission of protons and space-time characterisation in heavy-ion collisions}}
    %------------------------------------------------
\end{entrylist}

%----------------------------------------------------------------------------------------
% SCHOOLS
%----------------------------------------------------------------------------------------
% \section{Workshops and Schools}

% \begin{entrylist}
%   \entry
%     {2017}
%     {CERN European School of High-Energy Physics}
%     {Evora, Portugal}
%     {\emph{The European School is targeted particularly at students in experimental High-Energy Physics who are in the final years of work towards their PhDs}}
%     %------------------------------------------------
%   \entry
%     {2016}
%     {International ATLAS SUSY Workshop}
%     {Brighton, UK}
%     {\emph{One-week long workshop about the definition of the strategies for the Supersymmetry searches at ATLAS}}
%     %------------------------------------------------
%   \entry
%     {2015}
%     {STFC High-Energy Physics Summer School}
%     {Lancaster, UK}
%     {\emph{A two-week long school on theoretical and phenomenological High-Energy Physics}}
%     %------------------------------------------------
%   \entry
%     {2013}
%     {WPCF 2013 - IX Workshop on Particle Correlations and Femtoscopy}
%     {Catania, Italy}
%     {\emph{A three-day long workshop on particle-particle correlations and “femtoscopy” in Nuclear and Particle Physics}}
% \end{entrylist}

%----------------------------------------------------------------------------------------
% AWARDS SECTION
%----------------------------------------------------------------------------------------
\section{Awards}

\begin{entrylist}
  \entry 
      {2018}
      {Fellowship}
      {University of Sussex, UK}
      {STFC Impact Acceleration Account}
      %------------------------------------------------
  \entry 
      {2018}
      {Grant}
      {University of Sussex in collaboration with Deckchair.com}
      {DISCnet}
      %------------------------------------------------

  \entry
      {2016}
      {Grant}
      {University of Sussex, UK}
      {Doctoral Overseas Conference Grant for Postgraduate Researchers}
      %------------------------------------------------

  \entry
      {2015}
      {Scholarship}
      {STFC, UK}
      {4-year STFC-funded PhD scholarship in collaboration with CERN}
      %------------------------------------------------

  \entry
      {2013}
      {Scholarship}
      {INFN-LNS, Italy}
      {1-year INFN-funded scholarship in collaboration with LNS}
      %------------------------------------------------
\end{entrylist}

%----------------------------------------------------------------------------------------
% PUBLICATIONS SECTION
%----------------------------------------------------------------------------------------
 \clearpage\section{Publications}
 \printbibsection{article}{Articles in peer-reviewed journals} % Print all articles from the bibliography
 \begin{refsection} % This is a custom heading for those references marked as "inproceedings" but not containing "keyword=france"
 \nocite{*}
     \printbibliography[type=inproceedings, title={International peer-reviewed conferences/proceedings}, heading=bibheading]
\end{refsection}
% \printbibsection{report}{Research reports} % Print all research reports from the bibliography
\raggedbottom
\end{document}
