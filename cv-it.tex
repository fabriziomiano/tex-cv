%!TEX program = lualatex
%!TEX encoding = UTF-8 Unicode
% Awesome CV LaTeX Template for CV/Resume
%
% This template has been downloaded from:
% https://github.com/posquit0/Awesome-CV
%
% Author:
% Claud D. Park <posquit0.bj@gmail.com>
% http://www.posquit0.com
%
% Template license:
% CC BY-SA 4.0 (https://creativecommons.org/licenses/by-sa/4.0/)
%


%-------------------------------------------------------------------------------
% CONFIGURATIONS
%-------------------------------------------------------------------------------
% A4 paper size by default, use 'letterpaper' for US letter
\documentclass[11pt, a4paper]{awesome-cv}

% Configure page margins with geometry
\geometry{left=1.5cm, top=.8cm, right=1.5cm, bottom=.3cm, footskip=.5cm}

% Specify the location of the included fonts
\fontdir[fonts/]

% Color for highlights
% Awesome Colors: awesome-emerald, awesome-skyblue, awesome-red, awesome-pink, awesome-orange awesome-nephritis, awesome-concrete, awesome-darknight
\colorlet{awesome}{awesome-skyblue}
% Uncomment if you would like to specify your own color
% \definecolor{awesome}{HTML}{CA63A8}

% Colors for text
% Uncomment if you would like to specify your own color
% \definecolor{darktext}{HTML}{414141}
% \definecolor{text}{HTML}{333333}
% \definecolor{graytext}{HTML}{5D5D5D}
% \definecolor{lighttext}{HTML}{999999}

% Set false if you don't want to highlight section with awesome color
\setbool{acvSectionColorHighlight}{false}

% If you would like to change the social information separator from a pipe (|) to something else
\renewcommand{\acvHeaderSocialSep}{\quad\textbar\quad}

%-------------------------------------------------------------------------------
%	PERSONAL INFORMATION
%	Comment any of the lines below if they are not required
%-------------------------------------------------------------------------------
% Available options: circle|rectangle,edge/noedge,left/right
\photo[circle,edge,left]{./profile/profile.jpg}
\name{Fabrizio}{Miano}
\position{Data Science {\enskip\cdotp\enskip} BI Analytics {\enskip\cdotp\enskip} Cloud Computing}
\address{Catania, Italia}
\mobile{(+39) 333 60 66 844}
\email{fabriziomiano@gmail.com}
\linkedin{fabriziomiano}
\github{fabriziomiano}

\usepackage[
  backend=biber,
  bibencoding=utf8
  ]{biblatex}
\addbibresource{bibliography.bib}
%-------------------------------------------------------------------------------
\begin{document}
\sloppy
% Print the header with above personal informations
% Give optional argument to change alignment(C: center, L: left, R: right)
\makecvheader

% Print the footer with 3 arguments(<left>, <center>, <right>)
% Leave any of these blank if they are not needed
\makecvfooter
  {}
  {}
  {}

\makeatletter
\renewcommand\@biblabel[1]{}
\makeatother
%-------------------------------------------------------------------------------
%	CV/RESUME CONTENT
%	Each section is imported separately, open each file in turn to modify content
%-------------------------------------------------------------------------------
\cvsection{Overview}
	
In qualità di Data Scientist sviluppo algoritmi di machine learning (ML) ed intelligenza artificiale (AI) per Leonardo, in ambiti quali energy market, pubblica amministrazione e sicurezza, con focus su data crawling, computer vision, text mining, e natural language processing (NLP). In tali progetti oltre alle capacità tecniche era richiesta anche esperienza di gestione team e coordinamento progetti/risorse dal punto di vista di timing e budget. Precedentemente ho lavorato come Research Fellow in Data Science alla University of Sussex, dove ho lavorato allo sviluppo di un algoritmo di machine learning per la classificazione di immagini di tramonti e, inoltre, di video di testimonianze in tribunale come veritiere o ingannevoli. 

Nel 2018 mi sono laureato PhD in Experimental Particle Physics alla University of Sussex, dove ho prodotto diversi articoli scientifici riguardanti analisi di enormi quantità di dati, raccolti dall'esperimento ATLAS al LHC (CERN) di Ginevra, tramite codice custom Python e C++. Durante il corso di dottorato sono stato invitato a conferenze nazionali (Regno Unito), ed internazionali (Stati Uniti), dove ho dato diversi talk. Inoltre, ho partecipato ad attività di divulgazione scientifica.

%----------------------------------------------------------------------------------------
% WORK EXPERIENCE SECTION
%----------------------------------------------------------------------------------------

% \cvsection{Experience}
\cvsection{Esperienza aziendale}
  \begin{cventries}
    \cventry
    {Business Intelligence Consultant}
    {Experis IT}    
    {Catania, Italy}
    {Sep 2020 -- Now}
    {
      \begin{cvitems}
        \item {
          Analisi di customers demand e forecast con metodologia Agile tramite tecnologia Microsoft Azure: 
          MS SQL Server, Data Factory, Data Warehouse, Data Lake, Analysis Services, PowerBI
        }
      \end{cvitems}
    }

    \cventry
    {ICT Specialist / PM}
    {R2M Solution}    
    {Catania, Italy}
    {Jan 2020 -- Now}
    {
      \begin{cvitems}
        \item {Full-stack development di un ERP Custom (Frappé}
        \item {System administrator di web solutions on-premises e su cloud: Google cloud, AWS, Azure ed Heroku}
        \item {Attività tecniche in progetti di ricerca nei domini di robotica, intelligenza artificiale, big data, energy saving, ed IoT}
        \item {Soluzioni IT ed help desk}
      \end{cvitems}
    }

    \cventry
    {Data Scientist / Backend Developer}
    {Leonardo}    
    {Catania, Italia}
    {Gen 2019 -- Gen 2020}
    {
      \begin{cvitems}
        \item {Data crawling: sviluppo di un crawler scritto in Python per la raccolta dati da social ne}
        \item {Sistemi di raccomandazione: sviluppo di un sistema di raccomandazione per user-job matching}
        \item {Computer Vision: sviluppo di webapp per real-time object detection usando YOLO}
        \item {NLP: sviluppo di un tool per l'estrazione di informazioni da PDF; integrazione con RESTful API (Python, Flask)}
        \item {Social-network Analysis: named-entity extraction e sentiment analysis su commenti a post su Facebook; integrazione con RESTful API (Python, Flask)}
        \item {Creazione e gestione microservizi su portale Azure (Docker ed Azure Container Registry)}
      \end{cvitems}
    }

    \cventry
    {Junior Data Scientist}
    {Deckchair.com}
    {Brighton, UK}
    {Giu -- Ago 2018}
    {
      \begin{cvitems}
        \item Sviluppo di software per la selezione di immagini interessanti all'interno di un dataset (Python, skimage);
        \item Design di AI per classificare immagini di cieli grigi e tramonti (Python, sklearn, Keras);
      \end{cvitems}
    }
  \end{cventries}

\cvsection{Esperienza Accademica}
  \begin{cventries}
    \cventry
    {Post-Doctoral Research Fellow in Data Science}
    {University of Sussex}
    {Brighton, UK}
    {Set -- Dic 2018}
    {
      \begin{cvitems}
        \item Sviluppo di modelli di machine learning per classificare video di testimonianze in tribunale come veritiere o false (Python, Keras, Tensorflow)
        \item Esplorazione di potenziali collaborazioni con altre aree di ricerca a University of Sussex quali, Evolutionary and Adaptive Systems Research Group, Industrial Informatics and Signal Processing Research Group, Sensor Technology Research Centre
      \end{cvitems}
    }
  \end{cventries}

  \begin{cventries}    
    \cventry
    {Dottorato di Ricerca}
    {University of Sussex / CERN}
    {Brighton, UK / Ginevra, Svizzera}
    {Nov 2014 -- Oct 2018}
    {
    \begin{cvitems}
      \item Sviluppo di codice Python per ottimizzazione multi-variabile signal-over-background 
      \item Modelling di diversi rumori di fondo, usando un custom C++ framework
      \item Stima data-driven di un rumore \emph{irriducibile} e valutazione delle incertezze teoriche (C++, Python, Excel)
      \item Partecipazione a sviluppo e commissioning di un algoritmo di ricostruzione del vertice della reazione (C++)
      \item Validazione pre-production di cambiamenti al framework
      \item Monitoring delle performance delle efficienze di tracking
      \item Sviluppo di codice Python per monitorare, e mostrare su una webpage (HTML), i risultati di nuove nightlies del codice.
    \end{cvitems}
    }
  \end{cventries}

  \begin{cventries}    
    \cventry
    {Progetto di ricerca - Laurea Magistrale}
    {Università degli Studi di Catania}
    {Catania, Italy}
    {Mar 2013 -- Jul 2014}
    {
    \begin{cvitems}
      \item Sviluppo e validazione codice C per simulare distribuzioni d'interesse fisico dell'emissione di protoni da una sorgente nucleare di data taglia geometrica e vita media. 
      \item Deduzione di taglia geometrica e vita media di una sorgente nucleare creata in collisioni tra ioni pesanti ad energie intermedie, usando dati del rivelatore LASSA del NSCL, Michigan, Stati Uniti.
    \end{cvitems}
    }
  \end{cventries}

%-------------------------------------------------------------------------------
%	SECTION TITLE
%-------------------------------------------------------------------------------


% %-------------------------------------------------------------------------------
% %	CONTENT
% %-------------------------------------------------------------------------------
% \begin{cventries}

% %---------------------------------------------------------
%   \cventry
%     {B.S. in Computer Science and Engineering} % Degree
%     {POSTECH(Pohang University of Science and Technology)} % Institution
%     {Pohang, S.Korea} % Location
%     {Mar. 2010 - Aug. 2017} % Date(s)
%     {
%       \begin{cvitems} % Description(s) bullet points
%         \item {Got a Chun Shin-Il Scholarship which is given to promising students in CSE Dept.}
%       \end{cvitems}
%     }

% %---------------------------------------------------------
% \end{cventries}


%----------------------------------------------------------------------------------------
% EDUCATION SECTION
%----------------------------------------------------------------------------------------
\cvsection{Formazione}

\begin{cventries}

  \cventry
      {PhD in Experimental Particle Physics}
      {University of Sussex / CERN}
      {Brighton, UK / Ginevra, Svizzera}
      {2015--2018}
      {
      \href{http://cds.cern.ch/record/2650559?ln=en}
      {Optimisation studies and data-driven background estimation in searches for the 
      supersymmetric partner of the top quark with the ATLAS Detector at the LHC}
      }
      %------------------------------------------------

  \cventry
      {Laurea Magistrale in Fisica}
      {Università degli Studi di Catania}
      {Catania, Italia}
      {2009--2014}
      {Emissione di protoni e caratterizzazione spazio-temporale in collisioni tra ioni pesanti}
      %------------------------------------------------
      
  \cventry
      {Laurea in Fisica}
      {Università degli Studi di Catania}
      {Catania, Italia}
      {2005--2009}
      {Spettri di neutroni da sorgenti standard Am-Be e Pu-Be}
      %------------------------------------------------
\end{cventries}

\newpage
%-------------------------------------------------------------------------------
%	SECTION TITLE
%-------------------------------------------------------------------------------
\cvsection{Skills}


%-------------------------------------------------------------------------------
%	CONTENT
%-------------------------------------------------------------------------------
\begin{cvskills}
  \cvskill{Data Science}{
    \texttt{sklearn, skimage, Keras, pandas, numpy, OpenCV, spaCy, 
      NLTK, librosa, selenium, BeautifulSoup, Flask, Jupyter}}
%---------------------------------------------------------
  \cvskill
    {Languages}
    {\texttt{Python, Bash, C/C++, HTML, LaTeX, Java, JavaScript}}
%---------------------------------------------------------
  \cvskill{Versioning}{Git, SVN}
%---------------------------------------------------------
  \cvskill{Database}{MongoDB, SQL}
%---------------------------------------------------------
  \cvskill
    {DevOps}
    {Docker, Azure: WebApp, Function App, CosmosDB, EventHub, Azure CLI}
%---------------------------------------------------------
  \cvskill
    {Office Suite}
    {Word, Excel, PowerPoint}
%---------------------------------------------------------
  \cvskill
    {OSs}
    {Linux, macOS, MS Windows}
%---------------------------------------------------------
  \cvskill
    {Teaching}
    {Mentore del gruppo di Data Science; Fisica Classica ed Elettromagnetismo; Proprietà della materia}
%---------------------------------------------------------
  \cvskill
    {Lingue}
    {Inglese (Fluente), Spagnolo (Fluente), Francese (Base), Italiano (Madrelingua)}
%---------------------------------------------------------
\end{cvskills}

%-------------------------------------------------------------------------------
%	SECTION TITLE
%-------------------------------------------------------------------------------
% \cvsection{Presentation}
% %-------------------------------------------------------------------------------
% %	CONTENT
% %-------------------------------------------------------------------------------
% \begin{cventries}

% %---------------------------------------------------------
%   \cventry
%     {Presenter for <Hosting Web Application for Free utilizing GitHub, Netlify and CloudFlare>} % Role
%     {DevFest Seoul by Google Developer Group Korea} % Event
%     {Seoul, S.Korea} % Location
%     {Nov. 2017} % Date(s)
%     {
%       \begin{cvitems} % Description(s)
%         \item {Introduced the history of web technology and the JAM stack which is for the modern web application development.}
%         \item {Introduced how to freely host the web application with high performance utilizing global CDN services.}
%       \end{cvitems}
%     }

% %---------------------------------------------------------
% \end{cventries}



%----------------------------------------------------------------------------------------
% COMMUNICATION SKILLS SECTION
%----------------------------------------------------------------------------------------
\cvsection{Public Speaking}

\begin{cventries}
  \cventry
    {``From Particle Physics to Computer Vision''}
    {DISCnet showcase}
    {Royal Society, Londra, UK}
    {Dic 2018}
    {10-min talk sul software sviluppato per Deckchair.com}
    %------------------------------------------------

  \cventry
    {``HLT tracking performance''}
    {HLT-UK Meeting}
    {University of Oxford, UK}
    {Set 2017}
    {15-min talk: I risultati delle performance del tracking dell'Inner Detector di ATLAS }
    %------------------------------------------------

  \cventry
    {
    \href{http://cds.cern.ch/record/2263055/files/ATL-PHYS-SLIDE-2017-220.pdf}{``Searches for direct production of third generation squarks with the ATLAS detector''}
    }
    {Phenomenology 2017}
    {Pittsburgh, PA, Stati Uniti}
    {Mag 2017}
    {15-min talk: I risultati delle ricerche di squarks di terza generazione per conto della collaborazione ATLAS}
    %------------------------------------------------

  \cventry
    {
    \href{https://indico.cern.ch/event/432527/contributions/1071672/attachments/1317672/1974610/FMiano-IDTrigger-ICHEP.pdf}
    {``The Design and performance of the ATLAS Inner Detector Trigger for Run 2 collisions at $\sqrt{s} = 13$ TeV''}
    }
    {International Conference on High Energhy Physics (ICHEP) 2016}
    {Chicago, IL, Stati Uniti}
    {Aug 2016}
    {Poster presentato per conto della collaborazione ATLAS: ``Results of the performance of the ATLAS Inner Detector''}
    %------------------------------------------------

  \cventry
    {``Direct pair production of the top squark in all-hadronic final states in \emph{pp} collisions with the ATLAS detector''}
    {STFC HEP Summer School}
    {Lancaster, UK}
    {2015}
    {Poster: ``preliminary results of the optimisation of the regions of interest for the search of the supersymmetric partner of the top quark''}
    %------------------------------------------------

\end{cventries}
% %-------------------------------------------------------------------------------
% %	SECTION TITLE
% %-------------------------------------------------------------------------------
\cvsection{Awards}

\begin{cvhonors}
  \cvhonor 
      {Post-doctoral Fellowship}
      {STFC Impact Acceleration Account}
      {Brighton, UK}
      {2018}
      %------------------------------------------------
  \cvhonor 
      {Grant}
      {DISCnet}
      {Brighton, UK}
      {2018}
      %------------------------------------------------
  \cvhonor
      {Doctoral grant}
      {Doctoral Overseas Conference Grant per Postgraduate Researchers}
      {Chicago, IL, USA}
      {2016}
      %------------------------------------------------
  \cvhonor
      {Borsa di Studio}
      {Borsa quadriennale in collaborazione col CERN}
      {CERN, Svizzera}
      {2015}
      %------------------------------------------------   
  \cvhonor
      {Borsa di studio}
      {Borsa di 1 anno per attività di ricerca, in collaborazione con LNS}
      {Catania, Italy}
      {2013}
      %------------------------------------------------
\end{cvhonors}
%-------------------------------------------------------------------------------
%	SECTION TITLE
%-------------------------------------------------------------------------------
\cvsection{Divulgazione scientifica}

\begin{cvhonors}
  \cvhonor 
      {\#ScienceOnBuses}
      {Partecipazione in riprese per avvicinare il pubblico alle attività di ricerca svolte al CERN.}
      {Brighton, UK}
      {Giu 2018}
      %------------------------------------------------
  \cvhonor 
      {HiSPARC}
      {Costruzione di uno scintillatore per rivelazione di raggi cosmici con studenti di scuola media}
      {Brighton, UK}
      {Nov 2017}
      %------------------------------------------------

  \cvhonor
      {CERN Master class}
      {Talk per un audience di studenti scolastici riguardo ``The life of a PhD student at CERN''}
      {CERN, Svizzera}
      {Mag 2016}
      %------------------------------------------------

  \cvhonor
      {Brighton Science Festival}
      {Coinvolto bambini in attività inerenti la Fisica delle particelle}
      {Brighton, UK}
      {Feb 2015}
\end{cvhonors}


\newpage
\cvsection{Pubblicazioni}
\newrefcontext[sorting=ynt]
 \printbibsection{inproceedings}{Proceedings/conferenze internazionali peer-reviewed} % Print all articles from the bibliography
\newrefcontext[sorting=nty]
 \printbibsection{thesis}{Tesi di Dottorato}%, heading=bibheading]
\newrefcontext[sorting=nty]
 \printbibsection{article}{Articoli in giornali peer-reviewed} % Print all articles from the bibliography
%-------------------------------------------------------------------------------
\end{document}
